\chapter{Расчетная часть}
\section{Исходная надежность системы}
	Исходная схема имеет основное соединение, тогда ее ВБР с учетом эксп. закона надежности может быть рассчитана:
	\begin{gather}
		P_{InitSys} = \prod\limits_{i=1}^{n}e^{-\lambda_{i} t} = e^{-\lambda t n} ;\label{eq1}\\
		P_{InitSysMatlab} = exp(-lmbd*numElem*t) = 0.05.
	\end{gather}
\section{Игра с параметрами системы}
	Исходя из формулы \ref{eq1} надежность системы можно повысить за счет 3ех параметров:
	\begin{enumerate}
		\item $\lambda$ -- интенсивность отказов;
		\item $t$ -- время эксплуатации;
		\item $n$ -- количество элементов в схеме.
	\end{enumerate}
	\begin{gather}
		\lambda = \lg(\frac{1}{P})\frac{1}{t n};\\
		t = \lg(\frac{1}{P})\frac{1}{\lambda n};\\
		n = \lg(\frac{1}{P})\frac{1}{\lambda t}.
	\end{gather}

	В соответствии с этим можно высчитать значения параметров для достижения необходимого уровня надежности $P_{min} = Pmin = 0.98$.
	\begin{gather*}
		lmbdDcrs = log(1/Pmin)/t/numElem;\\
		tDcrs = log(1/Pmin)/numElem/lmbd;\\
		numElemDcrs = log(1/Pmin)/t/lmbd.
	\end{gather*}
	И оценить их расхождения и исходными значениями
	\begin{gather*}
		lmbdRtn = lmbdDcrs/lmbd = 0.00673424;\\
		tdltT = t - tDcrs  = 993.266;\\
		dltElmnts = numElem - numElemDcrs = 993.266.\\
	\end{gather*}

	Максимальное изменение интенсивности возможно лишь на один порядок. Разница во времени и количестве элементов не возможно, так как всего 1000ч работы и 1000 элементов.

	Попробовав изменить все основные характеристики в 0.1 и 0.5 раз были полученные:
	\begin{gather*}
		P_{0.1} = e^{-0.1*\lambda 0.1*t 0.1*n} = 0.997004;\\
		P_{0.5} = e^{-0.5*\lambda 0.5*t 0.5*n} = 0.687289.
	\end{gather*}
	Первый вариант нам подходит, но уменьшение всего до 10\% плохое решение. Найдем такой коэффициент, умножив на который все части мы получим необходимый уровень надежности:
	\begin{gather*}
		P_{k} = e^{-k^{3}\lambda t n} = P_{min}.
	\end{gather*}
	Решая данную систему в MATLAB получим, что $k = 0.188841$ и параметры системы становятся:
	\begin{enumerate}
		\item $\lambda = 5.66523e-07$;
		\item $t = 188$ч;
		\item $n = 188$.
	\end{enumerate}
	Порядок интенсивности нас устраивает, но такое сильное изменение времени и количества элементов недопустимо.
\section{Резервирование}
\subsection{Общее горячие резервирование}
	Рассмотрим случай, когда необходимо зарезервировать всю систему сразу. Найдем необходимую кратность такой системы из форм.~\ref{eq1-1}. Решая в MATLAB уравнение:
	\begin{gather*}
		Pmin == 1 - (1 - exp(-lmbd*numElem*t))^{m+1};
	\end{gather*}
	Получаем кратность равную $76$, не экономично, но как вариант...
\subsection{Специальное горячие резервирование}
	Рассмотрим случай, когда необходимо зарезервировать все элементы сразу. Найдем необходимую кратность такой системы из форм.~\ref{eq1-2}. Решая в MATLAB уравнение:
	\begin{gather*}
		1 - Pmin^{1/numElem} ==  (1 - exp(-lmbd*t))^{m+1};
	\end{gather*}
	получим кратность $m = 0.860328$ меньше единицы, значит мы можем резервировать не все элементы, найдем тогда минимальное количество таких элементов решая систему в MATLAB:
	\begin{gather*}
		Pmin == exp(-lmbd*t*(numElem - m))*(1 - (1 - exp(-lmbd*t))^{2})^{m};
	\end{gather*}
	и получим, что минимальное количество элементов, которое нужно зарезервировать $n = 997$.

	Изучая данный метод на чувствительность к количеству элементов (например 500) и интенсивности отказов получим, что
	\begin{enumerate}
		\item Если $n = 500$ то необходимо зарезервировать $495$;
		\item $\lambda = 0.1\lambda$. то необходимо зарезервировать $933$;
		\item $t = 0.9t$. то необходимо зарезервировать $926$;
	\end{enumerate}

\subsection{Специальное холодное резервирование}
	Рассматривая аналогично холодное резервирование получим:
	\begin{enumerate}
		\item Если $n = 1000, m = 1$ то необходимо зарезервировать $995$;
		\item Если $n = 500$ то необходимо зарезервировать $495$;
		\item $\lambda = 0.1\lambda$. то необходимо зарезервировать $933$;
		\item $t = 0.9t$. то необходимо зарезервировать $926$;
	\end{enumerate}
