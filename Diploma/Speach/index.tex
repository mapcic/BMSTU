% Shut up!
\RequirePackage{silence}
\WarningFilter{caption}{Unsupported document class}
\WarningFilter{latexfont}{Font shape `PD1/cmr/m/n'}
\WarningFilter{latexfont}{Font shape `PU/cmr/m/n'}
\WarningFilter{latexfont}{Some font shapes}

\documentclass[utf8, 14pt]{G7-32}

\sloppy

% Нумерация формул, таблиц и картинок относительно глав.
\EqInChapter
\TableInChapter
\PicInChapter

% Гипертекстовое оглавление.
\usepackage[
  bookmarks=true,  colorlinks=true, unicode=true,
  urlcolor=black,  linkcolor=black, anchorcolor=black,
  citecolor=black, menucolor=black, filecolor=black,
]{hyperref}

% Times, если стоит texlive-scalable-cyrfonts.
%\IfFileExists{cyrtimes.sty} {
  %\usepackage{cyrtimes}
%}

\usepackage{graphicx}
\graphicspath{{assets/}}

% Поля.
\geometry{right=15mm, top=20mm, bottom=20mm, left=30mm}

\usepackage{tikz}
\usetikzlibrary{arrows, arrows.meta}

\usepackage{pgfplots}
\pgfplotsset{compat=1.12}

\usepackage{enumerate}
\usepackage{multirow}
\usepackage{paralist, array}
\usepackage{fancyvrb}
\usepackage{changepage}

% Центрирование подписей.
\usepackage[justification=centering]{caption}


\begin{document}
\frontmatter
\begin{center}
	\begin{LARGE}
		Речь
	\end{LARGE}\\
	 на ВКБ \textsc{Прохорова М.Д.}\\
	 \begin{large}
	 <<Моделирование термической деградации $GaAs$ гетероструктур>>
	 \end{large}
\end{center}

\section{0}
Здравствуйте, уважаемая Государственная Аттестационная Комиссия!

Я, Прохоров Максим, представляю вам свою выпускную квалификационную работу на тему <<Моделирование термической деградации $GaAs$ гетероструктур>>.

\section{1}
Электронные устройства на основе гетероструктур широко используются во многих областях человеческой деятельности. HEMT применяющийся в высокочастотных устройствах. Лазер с ДГС в проигрывателе компакт-дисков. Солнечные элементы с гетероструктурами широко используются как для кос- мических, так и для земных программ — космическая станция ”Мир” уже почти 10 лет использует солнечные элементы на основе AlGaAs-гетероструктур. А сегодня уже и Flash память на основе QD.
Все это возможно благодаря особенностям зонной структуры ГП. 

\section{2}
Таким образом работа приборов на основе ГС зависит от их зонной структуры, которое под воздействием внешних факторов изменяется -- деградиует.

Гетероструктуры из которых состоят приборы, имеют разными химические составы и соответственно зонную структуру и особенность строения ГП. Под воздействием внешней среды атомы различных пп взаимодиффундируеют через границу ГС и происходит деградация ГС и выход прибора из строя.

Из-за деградации ГС при проектировании прибора необходимо учитывать технологию производства и условия его дальнейшую эксплуатации. Для прогнозирование параметров прибора на основе ГС можно... 

\section{3}
Цель работы:
\begin{itemize}
    \item Разработка алгоритма прогнозирования деградации приборов на основе $GaAs$ гетероструктур.
\end{itemize}
Задачи работы:
\begin{itemize}
    \item Моделирование диффузионного размытия гетероструктур на основе $GaAs$ под действием градиента концентрации при фиксированной температуре;
    \item Моделирование токопереноса через гетероструктуру на основе $GaAs$;
    \item Разработка алгоритма деградации резонансно-туннельной гетероструктуры на основе $GaAs$.
\end{itemize}

\section{4}
Для решения дифференциальных уравнений и дальнейшего моделирования токопереноса через ГС и диффузионного размытия ГС я выбрал численный метод, который называется методом конечных разностей. Суть метода в аппроксимации производной конечными разностями. На слайде представлена аппроксимация первой производной справа и трех точечная центральная аппроксимация второй производной. Так же на слайде предствелны Графические представление КРсх для нестаци. ур. дифф. и уравнения шредингера. На данных схемах функия S зависит от координаты x и времени t, dx  и dt -- это шаги нашей схемы, которые соединяют соседние значения функции.

\section{5}
Для решения нестац уравнения диффузии рассмотрено 3 КРсх. Явная схема, которое связывает значение функции в будущем с тремя значениями в настоящем. Данная схема является самой простой и удобна для вывода ГУ, но она имеет ограничение на сходимость, что сильно ограничивает применимость данной схемы.
Конечно разностная схема связывающая три значения функции в настоящем с одним в прошлом, называется неявно или BTHS. Метод не нагляден, но не имеет ограничений на сходимость в отличии от явной схемы.
Более точным и не имеющим ограничения на сходимость является метод Кранка-Никольсона, использует аппроксимацию первой производной слева, как и в неявной схеме, а в качестве аппроксимации второй производной по координате берется среднее из центральной трех точечной аппроксимации в момент настоящем и прошлом.

Для проверки схемы было взято аналитическое решение нестационарного ур. диф. из книги Advanced Engineering Mathematics, под авторством Эрвина Крэусзига. ГУ -- на краях системы концентрация значения равны 0 -- такую системы назывем открытой. НУ и аналитеское решения концентрации предствалены на слайде.
% \pagenumbering{gobble}
\end{document}