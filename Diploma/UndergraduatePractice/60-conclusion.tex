\Conclusion
В отчете рассматривается процесс моделирования деградации гетероструктур под действием температуры средствами среды \textsc{MatLab}. Моделирование термической деградации гетероструктур состоит из двух частей:
\begin{enumerate}
	\item Моделирование процессов структурной деградации гетероструктуры;
	\item Моделирование устройства на основе гетероструры.
\end{enumerate}
\textbf{Цель практики}:
\begin{enumerate}
	\item Изучение математического аппарата для моделирования процесса термической деградации;
	\item Изучение процессов деградации гетероструктур.
\end{enumerate}

\textbf{Задача практики}:
\begin{enumerate}
	\item Моделирование процессов термической диффузии;
	\item Моделирование токопереноса;
	\item Исследование влияния основных параметров РТГС на ВАХ;
	\item Моделирование термической деградации ВАХ РТГС на основе $AlGaAs$.
\end{enumerate}

\textbf{Индивидуальное задание}: Разработка алгоритма и программы, позволящей:
\begin{enumerate}
	\item Моделирование диффузии в твердых растворов в гетероструктурах;
	\item Моделирование диффузии в примеси;
	\item Моделирование расплытия потенциального рельефа гетероструктуры;
	\item Моделирование основных параметров РТГС на основе $AlGaAs$.
\end{enumerate}


В работе был рассмотрен математический аппарат для моделирования:
\begin{itemize}
	\item Токопереноса;
	\item Не стационарного уравнения диффузии.
\end{itemize}
Так же были написаны программы, которые рассчитывали ВАХ через РТГС и моделировали диффузию атомов $Al$ и $Si$ в ГС на основе $AlGaAs$.

Была рассмотрена РТГС на основе $AlGaAs$ и исследованы влияния ее основных параметров на ВАХ:
\begin{itemize}
	\item Глубина ПЯ;
	\item Ширина ПЯ;
	\item Ширина ПБ;
	\item Ширина спейсеров.
\end{itemize}

На основе полученных данных была исследована термическая деградация РТГС и выявлено, что доминирующим фактором, влияющим на деградацию ВАХ РТГС является диффузия донорной примеси из приконтактных областей.