\section{Метод конечных разностей для самосогласованного решение системы Шредингера--Пуассона}
Для решения системы Шредингера-Пуассона (\ref{eq:P-S}) применяют метод Гумеля. Метод Гумеля заключается в учете характера зависимости концентрации электронов от самосогласованного потенциала ($V_{S}$):
\begin{gather}
	n(z) = \frac{2^{1/2}m^{3/2}k_{B}T}{(2\pi)^{2}\hbar^{3}} \exp \frac{E_{F} - E_{c} +eV_{s}(z)}{k_{B}T} = n_{0}\exp\frac{V_{S}(z)}{V_{ref}};\\
	n_{0} = \frac{2^{1/2}m^{3/2}k_{B}T}{(2\pi)^{2}\hbar^{3}} \exp \frac{E_{F} - E_{c} }{k_{B}T};\\
	V_{ref} = \frac{k_{B}T}{e},
\end{gather}
\begin{conditions}
	$V_{s}$ & самосогласованный потенциал;\\
	$V_{ref}$ & опорный потенциал;\\
	$m$ & эффективная масса электрона.
\end{conditions}
Тогда связь <<старого>> и <<нового>> потенциала:
\begin{gather}
	\begin{cases}
		n_{old} = n_{0}\exp\frac{V_{old}}{V_{ref}};\\
		n_{new} = n_{0}\exp\frac{V_{new}}{V_{ref}};
	\end{cases}\\
	\label{eq:nNewOld}
	n_{new} = n_{old}\exp\frac{V_{new} - V_{old}}{V_{ref}};
\end{gather}

Подставляя (\ref{eq:nNewOld}) в (\ref{eq:P-S}):
\begin{equation}
 	\frac{d}{dz}\varepsilon(z)\frac{d}{z}V_{new} = \frac{e}{\varepsilon_{0}}\bigg[n_{old}\exp\bigg( \frac{V_{new} - V_{old}}{V_{ref}} \bigg) - N_{D}(z)\bigg];
\end{equation}

Так как $\frac{V_{new} - V_{old}}{V_{ref}}$ мало, используя разложение в ряд Маклорена $\exp(x) \approx 1 + x$, получим:
\begin{equation}
	\frac{d}{dz}\varepsilon(z)\frac{d}{z}V_{new} - n_{old}\frac{eV_{new}}{\varepsilon_{0}V_{ref}} = \frac{e}{\varepsilon_{0}}\bigg[n_{old}\bigg( 1 - \frac{V_{old}}{V_{ref}} \bigg) - N_{D}(z)\bigg];
\end{equation}

Применяя метод конечных разностей, получим:
\begin{equation}
	\begin{cases}
		a^{i}V_{new}^{i-1} + b^{i}V_{new}^{i} + c^{i}V^{i+1}_{new} = \frac{e\Delta^{2}}{\varepsilon_{0}\varepsilon^{i}}\bigg( n_{old}^{i}\bigg( 1 - \frac{V_{old}^{i}}{V_{ref}} \bigg) - N_{D}^{i} \bigg);\\
		a^{i} = 1;\\
		b^{i} = -1 - \frac{\varepsilon^{i+1}}{\varepsilon^{i}} -\frac{e\Delta^{2}n_{old}^{i}}{\varepsilon_{0} V_{ref}\varepsilon_{i}};\\
		c^{i} = \frac{\varepsilon^{i+1}}{\varepsilon^{i}};
	\end{cases}
\end{equation}
Граничные условия:
\begin{equation}
	V^{1}_{new} = 0;
	V^{N}_{new} = V;
\end{equation}

И продолжаем рассчитывать эту систему до достижения максимальной погрешности $\epsilon_{V}$, если выполняется условие:
\begin{equation}
 	\max(|V_{new}^{i} - V_{old}^{i}|) < \epsilon_{V},
\end{equation} 
процесс итерации останавливается.