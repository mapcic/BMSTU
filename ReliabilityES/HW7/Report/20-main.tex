\chapter{Теоретическая часть}
\section{Надёжность технических систем}

	Техническая система --- объект, представляющий собой множество взаимосвязанных элементов, рассматриваемых в определённом контексте как единое целое и отделённых от окружающей среды.
	
	Технический элемент --- объект, для которого в рамках данного рассмотрения не выделяются составные части.
	
	Структурная схема надёжности (ССН) --- логическое и графическое представление объекта, отображающее, каким образом безотказность его блоков и их сочетаний влияют на безотказность объекта. 

	ССН представляет собой условную запись работоспособного состояния системы через работоспособность элементов этой системы. ССН может быть задана:
	\begin{enumerate}
		\item аналитически;
		\item графически. 
	\end{enumerate}
	ССН должна:
	\begin{enumerate}
		\item иметь физический смысл;
		\item достаточно просто описывать работоспособность системы; 
		\item поддаваться алгоритмизации. 
	\end{enumerate}
	Виды ССН: 
	\begin{enumerate}
		\item ССН с основным соединением элементов (с последовательным соединением); 
		\item ССН с резервным соединением элементов (присутствуют резервирующие элементы).
	\end{enumerate}
\section{Виды резервирования}
	Выделяется 5 основных видов резервирования (целых крайностей):
	\begin{enumerate}
		\item Общее горячие резервирование;
		\item Специальное горячие резервирование;
		\item Общее холодное резервирование;
		\item Специальное холодное резервирование;
		\item Скользящие холодное резервирование;
	\end{enumerate}
\subsection{Общее горячие резервирование}
	\begin{equation}\label{eq1-1}
		P_{system}(t) = 1 - \prod\limits_{i = 1}^{ m + 1 }\Big( 1 - \prod\limits_{j = 1}^{ N }P_{ij}(t) \Big),
	\end{equation}
	\begin{conditions}
	  $m$ & кратность резервирования,\\
	  $N$ & количество элементов,\\
	  $P_{ij}$ & ВБР $ij$ элемента $N$.
	\end{conditions}
\subsection{Специальное горячие резервирование}
	\begin{equation}\label{eq1-2}
		P_{system}(t) = \prod\limits_{j = 1}^{ N } \Big(1 - \prod\limits_{i = 1}^{ m + 1 }\big( 1 - P_{ij}(t) \big) \Big),
	\end{equation}
	\begin{conditions}
	  $m$ & кратность резервирования,\\
	  $N$ & количество элементов,\\
	  $P_{ij}$ & ВБР $ij$ элемента $N$.
	\end{conditions}
\subsection{Специальное холодное резервирование}
	\begin{equation}\label{eq1-3}
		P_{system}(t) = \prod\limits_{i = 1}^{ N }\Big( e^{-\lambda_{i} t}\sum\limits_{j = 0}^{ m }\frac{(\lambda_{j} t)^{j}}{j!} \Big),
	\end{equation}
	\begin{conditions}
	  $m$ & кратность резервирования,\\
	  $N$ & количество элементов,\\
	  $\lambda_{i}$ & интенсивность отказа $i$-ого элемента.
	\end{conditions}
