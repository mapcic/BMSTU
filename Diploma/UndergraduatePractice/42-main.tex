\section{Исследование параметров спейсеров}
Спейсер предназначен для отделения высоколегированной области вырожденного полупроводника от активной. Диффундируя донорная примесь изменяла бы зонную структуру активной (квантовой) области. Так же спейсер препятствует накоплению заряда вблизи барьеров, что влияет на пиковое напряжение и ток.
\subsection{Исследование влияния размеров спейсера}
Рассмотрим спейсеры: 
\begin{itemize}
	\item 3 монослоя;
	\item 7 монослоя;
	\item 10 монослоя;
	\item 15 монослоя.
\end{itemize}
\subsubsection{Прозрачность РТГС}

\begin{figure}[h]
	\centering
	\includegraphics[width=\linewidth]{qslt.png}
	\caption{Прозрачность РТГС при различной ширине спейсеров}
	\label{fig:qslt}
\end{figure}

Прозрачность гетероструктуры изменяется незначительно.

\subsubsection{ВАХ РТГС}
Увеличение размеров спейсера ведет к незначительному (порядок не изменяется), а пиковое напряжение не смещается.

\begin{figure}[h!]
	\centering
	\includegraphics[width=0.9\linewidth]{qslj.png}
	\caption{ВАХ РТГС при различной ширине спейсеров}
	\label{fig:qslj}
\end{figure}

\subsection{Вывод}
Уменьшение длины спейсера приводит к увеличению величины пикового тока, при этом пиковое напряжение не смещается и остается постоянным.
