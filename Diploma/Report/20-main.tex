\chapter{Теоретическая часть}
\section{Деградация}
Деградация~--- процесс ухудшения характеристик какого-либо объекта c течением времени.

Изучая деградацию гетероструктур (ГС) рассматривают следующие параметры:
\begin{itemize}
	\item Вольт-амперная характеристика (ВАХ);
	\item Высота потенциального барьера (ПБ);
	\item Ширина потенциального барьера;
	\item Ширина потенциальной ямы (ПЯ);
	\item Т.д...
\end{itemize}

ГС используют для построения резонансно-туннельный диод (РТД), квантовых точек (КТ), транзисторов с высокой подвижностью электронов (HEMT) и так далее. 

Химический состав ГС определяет ее зонную структуру, из чего вытекают особенности работы тех или иных устройств на ГС.

Одна из причин деградации ВАХ ГС~--- диффузионное размытие профиля дна зоны проводимости ($E_{c}$). Некоторые факторы, от которых зависит диффузионное размытие:
\begin{itemize}
	\item Химический состав; 
	\item Температура; 
	\item Время. 
\end{itemize}

Диффузионное размытие описывается с помощью законов Фика.

% \section{Полупроводники}
% Полупроводники (п/п)- широкий класс веществ, в которых концентрация подвижных носителей заряда значительно ниже, чем концентрация атомов, и может изменяться под влиянием температуры. освещения или относительно малого количества примесей.

% Эффективная плотность состояний в зоне проводимости (ЗП)~\cite{MFTIne}:
% \begin{equation}
% 	N_{c} = 2\Big[ \frac{2\pi m_{e}^{\ast}k_{B}T}{h^{2}} \Big]
% 	^{\frac{3}{2}},
% \end{equation}
% \begin{conditions}
% 	$m_{e}^{\ast}$ & эффективная масса электрона;\\
% 	$k_{B}$ & константа Больцмана;\\
% 	$h$ & постоянная Планка;\\
% 	$T$ & температура.
% \end{conditions}

% Эффективная плотность состояний в валентной зоне (ВЗ)~\cite{MFTIne}:
% \begin{equation}
% 	N_{v} = 2\Big[ \frac{2\pi m_{h}^{\ast}k_{B}T}{h^{2}} \Big]
% 	^{\frac{3}{2}},
% \end{equation}
% \begin{conditions}
% 	$m_{h}^{\ast}$ & эффективная масса дырки.
% \end{conditions}

% \subsection{Собственные полупроводники}
% Собственный полупроводник (п/п i-типа)~--- это чистый полупроводник, содержание посторонних примесей в котором не превышает $10^{−8} … 10^{−9}\%$.

% Концентрация собственных носителей заряда в ЗП~\cite{MFTIne}:
% \begin{equation}
% 	n_{i} = \sqrt{N_{c}N_{v}}\exp\!\bigg[ - \frac{E_{g}}{2k_{B}T} \bigg],
% \end{equation}
% \begin{conditions}
% 	$E_{g}$ & ширина запрещенной зоны (ЗЗ) п/п.
% \end{conditions}

% \subsection{Примесные полупроводники}
% Примесный полупроводник - это полупроводник электрофизические свойства которого определяются, в основном, примесями других химических элементов.

% Концентрация электронов в ЗП примесного п/п ~\cite{MFTIne}:
% \begin{equation}
% 	n = \frac{N_{D}}{2}\bigg( 2 + \frac{1}{2}\bigg( \frac{2n_{i}}{N_{D}} \bigg)^{2} \bigg),
% \end{equation}
% \begin{conditions}
% 	$N_{D}$ & концентрация атомов легирующей примеси.
% \end{conditions}