\chapter{Исследование параметров РТГС на основе $Al_{x}Ga_{1-x}As$}
\section{Исследование параметров ямы}
Энергетический спектр в бесконечно глубокой потенциальной яме:
\begin{gather}
	\label{eq:En}
	E_{n} = \frac{\pi^{2}\hbar^{2}n^{2}}{2mL^{2}};\\
	\Delta E_{n} = E_{n+1} - E_{n} = \frac{\pi^{2}\hbar^{2}}{2mL^{2}}(2n + 1);\\
	\label{eq:dEn1}
	\Delta E_{1} = \min(\Delta E_{n}) = \frac{3\pi^{2}\hbar^{2}}{2mL^{2}},
\end{gather}
\begin{conditions}
	$E_{n}$ & энергия $n$-ого связного состояния;\\
	$\hbar$ & постоянная Дирака;\\
	$n$ & номер ($1,\,,2,\dots$) связного состояния;\\
	$L$ & ширина ямы.
\end{conditions}

Из зависимостей~\ref{eq:En},~\ref{eq:dEn1}  видно, что с увеличением ширины ямы, энергия основного состояния уменьшается ($n = 1$), так же, как и минимальное расстояние, между энергетическими уровнями. Условие размерного квантования для потенциальной ямы:
\begin{equation}
	\Delta E_{1} \gg 3k_{B}T.
\end{equation}

В случаи ямы с конечной высотой, энергия основного и остальных состояний понижается, при этом минимальная высота барьера ограничивает состояние с максимальной энергией. В потенциальной яме конечной высоты всегда будет хотя бы одно связное состояние.

\subsection{Исследование глубины ямы}
Варьирую процентное содержание $Al$ в $Al_{x}Ga_{1-x}As$, можно изменять высоту барьеров и глубину ямы соответственно. Рассмотрим вольт-амперную характеристику и прозрачность резонансно-туннельной структуры.

Так как в ходе деградации ГС высота барьеров будет уменьшаться, так как атомы $Al$ будут диффундировать в структуру из барьеров, рассмотрим РГТС (рис.~\ref{img:rtgs}) с высотой потенциальных барьеров:
\begin{enumerate}
	\item $1.0 eV$;
	\item $0.7 eV$;
	\item $0.5 eV$;
	\item $0.3 eV$;
\end{enumerate}


\subsubsection{Прозрачность РТГС}
\subsubsection{ВАХ РТГС}

\subsection{Исследование ширины ямы}
\subsubsection{ВАХ РТГС}
\subsubsection{Прозрачность РТГС}