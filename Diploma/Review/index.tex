% Shut up!
\RequirePackage{silence}
\WarningFilter{caption}{Unsupported document class}
\WarningFilter{latexfont}{Font shape `PD1/cmr/m/n'}
\WarningFilter{latexfont}{Font shape `PU/cmr/m/n'}
\WarningFilter{latexfont}{Some font shapes}

\documentclass[utf8, 14pt]{G7-32}

\include{preamble.inc}

\begin{document}
\frontmatter
\begin{center}
	\begin{LARGE}
		Рецензия
	\end{LARGE}\\
	 на дипломную работу \textsc{Прохорова М.Д.}\\
	 \begin{large}
	 <<Моделирование термической деградации гетероструктур>>
	 \end{large}
\end{center}

Работа посвящена актуальной проблеме повышения надежности гетероструктур (ГС), а именно моделированию термической деградации ГС и направлена на дальнейшее развитие и обобщение модели деградации ГС.

% Проектирование сложных приборов на основе ГС приводит к необходимости анализа наиболее подверженных деградации областей.

В дипломной работе Прохоров М.Д. систематизировал теоретический материал для создания модели термической деградации устройства на основе ГС и его дальнейшего анализа. Получены численные модели для процесса диффузии частиц под действием градиента концентрации при фиксированной температуре и токопереноса через ГС.

В качестве исследуемого образца в работе рассмотрена туннельно-резонансная гетероструктура (РТГС) на основе твердого раствора $Al_{x}Ga_{1-x}As$. Материал для исследования является одним из наиболее распространенных для получения различных ГС. 

Изученное влияния отдельных параметров РТГС на вольт-амперную характеристику (ВАХ) позволило при дальнейшем моделировании диффузии под действием градиента концентрации при заданной температуре РТГС, выявить доминирующие факторы и показать тенденцию термической деградации ВАХ рассмотренной РТГС.

Работа Прохорова М.Д. соответствует требования, предъявляемым к квалификационной работе бакалавра, к защите рекомендована и заслуживает оценки <<отлично>>. Прохоров М.Д. заслуживает присвоения квалификации бакалавра по направлению <<Наноинженерия>>.\\[2.5cm]
\noindent
\begin{minipage}{0.5\textwidth}
    \begin{flushleft} \large
    	Кандидат Технических Наук,\\
    	Ведущий Научный Сотрудник\\
    	АО НИИ <<ТП>>
    \end{flushleft}
\end{minipage}
~
\begin{minipage}{0.5\textwidth}
    \begin{flushright} \large
    	Сафронов В.В.
    \end{flushright}
\end{minipage}
\pagenumbering{gobble}
\end{document}