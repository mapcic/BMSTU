\Conclusion
В работе был рассмотрен математический аппарат для моделирования:
\begin{itemize}
	\item Токопереноса;
	\item Не стационарного уравнения диффузии.
\end{itemize}
Так же были написаны программы, которые рассчитывали ВАХ через РТГС и моделировали диффузию атомов $Al$ и $Si$ в ГС на основе $AlGaAs$.

Была рассмотрена РТГС на основе $AlGaAs$ и исследованы влияния ее основных параметров на ВАХ:
\begin{itemize}
	\item Глубина ПЯ;
	\item Ширина ПЯ;
	\item Ширина ПБ;
	\item Ширина спейсеров.
\end{itemize}

На основе полученных данных была исследована термическая деградация РТГС и выявлено, что доминирующим фактором, влияющим на деградацию ВАХ РТГС является диффузия донорной примеси из приконтактных областей.