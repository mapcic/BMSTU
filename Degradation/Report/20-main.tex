\chapter{Теоретическая часть}
% \section{Деградация}
% Деградация~--- процесс ухудшения характеристик какого-либо объекта c течением времени.

% Изучая деградацию гетероструктур (ГС) рассматривают следующие параметры:
% \begin{itemize}
% 	\item Вольт-амперная характеристика (ВАХ);
% 	\item Высота потенциального барьера (ПБ);
% 	\item Ширина потенциального барьера;
% 	\item Ширина потенциальной ямы (ПЯ);
% 	\item Т.д...
% \end{itemize}

% ГС используют для построения резонансно-туннельный диод (РТД), квантовых точек (КТ), транзисторов с высокой подвижностью электронов (HEMT) и так далее. 

% Химический состав ГС определяет ее зонную структуру, из чего вытекают особенности работы тех или иных устройств на ГС.

% Одна из причин деградации ВАХ ГС~--- диффузионное размытие профиля дна зоны проводимости ($E_{c}$). Некоторые факторы, от которых зависит диффузионное размытие:
% \begin{itemize}
% 	\item Химический состав; 
% 	\item Температура; 
% 	\item Время. 
% \end{itemize}

% Диффузионное размытие описывается с помощью законов Фика.

% % \section{Полупроводники}
% % Полупроводники (п/п)- широкий класс веществ, в которых концентрация подвижных носителей заряда значительно ниже, чем концентрация атомов, и может изменяться под влиянием температуры. освещения или относительно малого количества примесей.

% % Эффективная плотность состояний в зоне проводимости (ЗП)~\cite{MFTIne}:
% % \begin{equation}
% % 	N_{c} = 2\Big[ \frac{2\pi m_{e}^{\ast}k_{B}T}{h^{2}} \Big]
% % 	^{\frac{3}{2}},
% % \end{equation}
% % \begin{conditions}
% % 	$m_{e}^{\ast}$ & эффективная масса электрона;\\
% % 	$k_{B}$ & константа Больцмана;\\
% % 	$h$ & постоянная Планка;\\
% % 	$T$ & температура.
% % \end{conditions}

% % Эффективная плотность состояний в валентной зоне (ВЗ)~\cite{MFTIne}:
% % \begin{equation}
% % 	N_{v} = 2\Big[ \frac{2\pi m_{h}^{\ast}k_{B}T}{h^{2}} \Big]
% % 	^{\frac{3}{2}},
% % \end{equation}
% % \begin{conditions}
% % 	$m_{h}^{\ast}$ & эффективная масса дырки.
% % \end{conditions}

% % \subsection{Собственные полупроводники}
% % Собственный полупроводник (п/п i-типа)~--- это чистый полупроводник, содержание посторонних примесей в котором не превышает $10^{−8} … 10^{−9}\%$.

% % Концентрация собственных носителей заряда в ЗП~\cite{MFTIne}:
% % \begin{equation}
% % 	n_{i} = \sqrt{N_{c}N_{v}}\exp\!\bigg[ - \frac{E_{g}}{2k_{B}T} \bigg],
% % \end{equation}
% % \begin{conditions}
% % 	$E_{g}$ & ширина запрещенной зоны (ЗЗ) п/п.
% % \end{conditions}

% % \subsection{Примесные полупроводники}
% % Примесный полупроводник - это полупроводник электрофизические свойства которого определяются, в основном, примесями других химических элементов.

% % Концентрация электронов в ЗП примесного п/п ~\cite{MFTIne}:
% % \begin{equation}
% % 	n = \frac{N_{D}}{2}\bigg( 2 + \frac{1}{2}\bigg( \frac{2n_{i}}{N_{D}} \bigg)^{2} \bigg),
% % \end{equation}
% % \begin{conditions}
% % 	$N_{D}$ & концентрация атомов легирующей примеси.
% % \end{conditions}


\section{Гетероструктура}
Гетероструктура~--- полупроводниковая структура с несколькими гетеропереходами (ГП). 

% ГС получили широкое распространение из-за возможности, изменяя на границах ГС ширину запрещённой зоны, управлять движением носителей заряда.

Гетеропереход~--- контакт двух различных по химическому составу монокристаллических или аморфных полупроводников.

% ГП может образоваться между полупроводниками с абсолютно одинаковыми постоянными решетки, образующими монолитный, однородный в контакте, кристалл.
Наиболее распротраненные полупроводники для составления ГС:
\begin{enumerate}
	\item $GaAs$--$AlAs$;
	\item $GaN$--$AlN$;
	\item $GaSb$--$AlSb$--$InAs$;
	\item $GaAs$--$Ge$.
\end{enumerate}

\subsection{Зонная диаграмма гетероперехода}
Для построения зонной диаграммы необходимо знать ширину запрещенной зоны ($E_{g}$) и положение уровня Ферми ($E_{F}$) для контактируемых полупроводников.

\begin{figure}[h]
  \centering
  \includegraphics[width=.5\linewidth]{assets/Eg}
  \caption{Зонная диаграмма перехода между полупроводниками с различными $E_{g}$}
  \label{img:2.0.0}
\end{figure}

Одна из самых распространенных ГС~--- это ГС на основе твердого раствора $Al_{x}Ga_{1−x}As$, где $x$ -- это доля замещения.
Основные характеристики $Al_{x}Ga_{1−x}As$:

\begin{center}
  \begin{longtable}{|c|c|}
    \caption{Основные параметры $Al_{x}Ga_{1−x}As$}
    \label{tab:2.0.0}
    \\ \hline
    Параметр & $Al_{x}Ga_{1−x}As$ \\
    \hline \endfirsthead
    \subcaption{Продолжение таблицы~\ref{tab:2.0.0}}
    \\ \hline \endhead
    \hline \subcaption{Продолжение на след. стр.}
    \endfoot
    \hline \endlastfoot
	Кристаллическая структура& Типа цинковой обманки \\ \hline
	Постоянная решетки $a[nm]$  & $0.56533+0.00078x$ \\ \hline
	$E_{g}^{\Gamma}[eV],\, x < 0.45$    & $1.424+1.247x$ \\ \hline
	$E_{g}^{\Gamma}[eV],\, x > 0.45$    & $1.656+0.215x+0.143x^{2}$ \\ \hline
	% $\Delta E_{c}^{\Gamma}[eV],\, x < 0.45$    & $0.773x$ \\ \hline
	% $\Delta E_{c}^{\Gamma}[eV],\, x > 0.45$    & $0.232-0.259x+1.147x^{2}$ \\ \hline
	$m_{e}^{\Gamma}$    & $0.067+0.083x$ \\ \hline
	$m_{lh}$    & $0.082+0.071x$ \\ \hline
	$N_{atoms}[1/sm^{-3}]$    & $(4.42-0.17x)10^{22}$
  \end{longtable}
\end{center}

Следует также принимать во внимание, что полупроводники могут иметь минимумы зоны проводимости в разных точках зоны Брюллиена. К примеру, минимум зоны проводимости $GaAs$ находится в точке $\Gamma$, в то время как наименьший минимум в $AlAs$ близок к точке $X$. Таким образом, природа низшего минимума зоны проводимости меняется при изменении доли $Al$ в твердом растворе $Al_{x}Ga_{1−x}As$. Низший минимум в $Al_{x}Ga_{1−x}As$ изменяется от прямого расположения (минимум в $\Gamma$) зон до непрямой зонной структуры (минимум в $Х$) при содержании $Al \approx 45\%$. Обычно твердый раствор $Al_{x}Ga_{1−x}As$ получают с долей $Al$, меньше $0.45$, чтобы получить прямое расположение зон.

Рассматривая ГП $i\!-\!GaAs/i\!-\!Al_{x}Ga_{1−x}As$, при $x < 0.45$, получим высоту потенциальной ступеньки в зоне проводимости ($E_{c2} - E_{c1}$) $U \approx 1.247*x\, eV$.
% \section{Конкретная структура на основе $Al_{x}Ga_{1−x}As$}

% Более подробно рассмотрим гетероструктуру на основе $Al_{x}Ga_{1−x}As$. Возьмем $GaAl$/$AlAs$/$GaAl$/$AlAs$/$GaAl$, с соответствующими размерами $8$/$4$/$6$/$4$/$8$ -- монослоев.