\section{Метод конечных разностей для решения одномерного нестационарного уравнения диффузии}
\subsection{Коэффициент диффузии не зависит от концентрации}
% Одномерное нестационарное уравнение диффузии, соответствующее второму закону Фика имеет вид:
% \begin{equation}\label{eq:difft}
% 	\frac{\delta}{\delta t} C = D\frac{\delta^{2}}{\delta x^2} C;
% \end{equation}

% Аппроксимация первой производной по времени в момент времени $t_{i}$ концентрации $C_{j}(t_{i}) = C^{i}_{j}$ в точке $j$:
% \begin{equation}\label{eq:diffdt1}
% 	\frac{\delta}{\delta t} C^{i}_{j} = \frac{C^{i+1}_{j} - C^{i}_{j}}{\Delta t};
% \end{equation}

% Аппроксимация первой производной по координате в момент времени $t_{i}$ концентрации $C_{j}(t_{i}) = C^{i}_{j}$ в точке $j$:
% \begin{equation}\label{eq:diffdx1}
% 	J^{i}_{j} = \frac{\delta}{\delta x} C^{i}_{j} = \frac{C^{i}_{j+1} - C^{i}_{j}}{\Delta x};
% \end{equation}

% Аппроксимация второй производной по координате в момент времени $t_{i}$ концентрации $C_{j}(t_{i}) = C^{i}_{j}$ в точке $j$:
% \begin{equation*}
% 	\frac{\delta^{2}}{\delta x^{2}} C^{i}_{j} = \frac{\delta}{\delta x}\bigg[ \frac{C^{i}_{j+1} - C^{i}_{j}}{\Delta x} \bigg] = \frac{ \frac{C^{i}_{j+1} - C^{i}_{j}}{\Delta x} - \frac{C^{i}_{j} - C^{i}_{j-1}}{\Delta x}}{\Delta x} = 
% \end{equation*}
% \begin{equation}\label{eq:diffdx2}
% 	= \frac{C^{i}_{j+1} - 2C^{i}_{j} + C^{i}_{j-1}}{\Delta x^2};
% \end{equation}

% Подставляя в (\ref{eq:difft}) аппроксимацию производных (\ref{eq:diffdt1}), (\ref{eq:diffdx2}), получим связь $C^{i+1}_{j}$ с $C^{i}_{j}$, т.е. изменение концентрации через $\Delta t$:

% \begin{equation}\label{eq:diffFD}
% 	C^{i+1}_{j} = \lambda C^{i}_{j-1} + (1 - 2\lambda)C^{i}_{j} + \lambda C^{i}_{j+1},
% \end{equation}
% где $\lambda = \frac{D\Delta t}{\Delta x^2}$~---  связь коэффициента диффузии и шагов по сетке времени и координаты.

% Уравнение (\ref{eq:diffFD}) справедливо для всех не крайних точек конечно разностной схемы, при коэффициенте диффузии не зависящем от концентрации.

% Выделим два граничных приближения для концентрации:
% \begin{enumerate}
% 	\item <<Закрытая система>> ~--- концентрация на границе не изменяется ($J_{0}^{i} = 0$, $J_{N+1}^{i} = 0$);
% 	\item <<Открытая система>> ~--- поток частиц подходящий к границе равен потоку уходящих частиц ($J_{0}^{i} = J_{1}^{i}$, $J_{N}^{i} = J_{N+1}^{i}$).
% \end{enumerate}

% \begin{figure}
%     \centering
%     \begin{subfigure}[b]{0.45\textwidth}
%         \includegraphics[width=\textwidth]{CD}
%         \caption{A gull}
%         \label{fig:gull}
%     \end{subfigure}
%     \begin{subfigure}[b]{0.45\textwidth}
%         \includegraphics[width=\textwidth]{OD}
%         \caption{A mouse}
%         \label{fig:mouse}
%     \end{subfigure}
%     \caption{Pictures of animals}\label{fig:animals}
% \end{figure}

% \noindent
% \begin{minipage}{0.4\textwidth}
% 	\centering
% 	\includegraphics[width=\linewidth]{assets/CD}
% 	а)
% \end{minipage}
% ~
% \begin{minipage}{0.6\textwidth}
% 	\centering
% 	\includegraphics[width=\linewidth]{assets/OD}
% 	б)
% \end{minipage}
% \begin{figure}
% 	\centering
% 	% \includegraphics[width=0.9\linewidth]{assets/CD}
% 	\caption{a) <<Закрытая>>, б) <<Открытая>> система диффузии}
% 	\label{fig:DS}
% \end{figure}



% \begin{figure}
% 	\centering
% 	\includegraphics[width=0.8\linewidth]{assets/OD}
% 	\caption{"Открытая" система диффузии}
% 	\label{fig:OD}
% \end{figure}


<<Закрытая система>> ~--- концентрация на границе не изменяется ($J_{0}^{i} = 0$, $J_{N+1}^{i} = 0$):
\begin{equation}
	\begin{cases}
		C^{i+1}_{1} = (1 - \lambda)C^{i}_{1} + \lambda C^{i}_{2};\\
		C^{i+1}_{j} = \lambda C^{i}_{j-1} + (1 - 2\lambda)C^{i}_{j} + \lambda C^{i}_{j+1},\,j \in [2,\,\dots,\,N-1];\\
		C^{i+1}_{N} = (1 - \lambda)C^{i}_{N} + \lambda C^{i}_{N-1};\\
		\lambda = D\frac{\Delta t}{\Delta x^{2}}.
	\end{cases}
\end{equation}

<<Открытая система>> ~--- поток частиц подходящий к границе равен потоку уходящих частиц ($J_{0}^{i} = J_{1}^{i}$, $J_{N}^{i} = J_{N+1}^{i}$):
\begin{equation}
	\label{eq:DDiffConst}
	\begin{cases}
		C^{i+1}_{1} = C^{i}_{1};\\
		C^{i+1}_{j} = \lambda C^{i}_{j-1} + (1 - 2\lambda)C^{i}_{j} + \lambda C^{i}_{j+1},\,j \in [2,\,\dots,\,N-1];\\
		C^{i+1}_{N} = C^{i}_{N};\\
		\lambda = D\frac{\Delta t}{\Delta x^{2}}.
	\end{cases}
\end{equation}

\subsection{Коэффициент диффузии зависит от концентрации}

% Если коэффициенте диффузии (D) зависит от концентрации, тогда уравнение диффузии принимает вид:

% \begin{equation}\label{eq:diffD(x)t}
% 	\frac{\delta}{\delta t} C = \frac{\delta}{\delta x}D\frac{\delta}{\delta x} C;
% \end{equation}

Проводя рассуждения аналогичные предыдущему параграфу получит конечно-разностную схему для открытой схемы:
\begin{equation}
	\begin{cases}
		C^{i+1}_{1} = C^{i}_{1};\\
		C^{i+1}_{j} = \lambda_{-}^{i} C^{i}_{j-1} + (1 - \lambda^{i}_{+} - \lambda^{i}_{-})C^{i}_{j} + \lambda^{i}_{+} C^{i}_{j+1},\,j \in [2,\,\dots,\,N-1];\\
		C^{i+1}_{N} = C^{i}_{N};\\
		\lambda^{i}_{+} = D_{j+}^{i}\frac{\Delta t}{\Delta x^{2}};\\
		\lambda^{i}_{-} = D_{j-}^{i}\frac{\Delta t}{\Delta x^{2}}.
	\end{cases}
\end{equation}
