\chapter{Теоретическая часть}
\section{Полупроводники}
Полупроводники (п/п)- широкий класс веществ, в которых концентрация подвижных носителей заряда значительно ниже, чем концентрация атомов, и может изменяться под влиянием температуры. освещения или относительно малого количества примесей.

Эффективная плотность состояний в зоне проводимости (ЗП)~\cite{MFTIne}:
\begin{equation}
	N_{c} = 2\Big[ \frac{2\pi m_{e}^{\ast}k_{B}T}{h^{2}} \Big]
	^{\frac{3}{2}},
\end{equation}
\begin{conditions}
	$m_{e}^{\ast}$ & эффективная масса электрона;\\
	$k_{B}$ & константа Больцмана;\\
	$h$ & постоянная Планка;\\
	$T$ & температура.
\end{conditions}

Эффективная плотность состояний в валентной зоне (ВЗ)~\cite{MFTIne}:
\begin{equation}
	N_{v} = 2\Big[ \frac{2\pi m_{h}^{\ast}k_{B}T}{h^{2}} \Big]
	^{\frac{3}{2}},
\end{equation}
\begin{conditions}
	$m_{h}^{\ast}$ & эффективная масса дырки.
\end{conditions}

\subsection{Собственные полупроводники}
Собственный полупроводник (п/п i-типа)~--- это чистый полупроводник, содержание посторонних примесей в котором не превышает $10^{−8} … 10^{−9}\%$.

Концентрация собственных носителей заряда в ЗП~\cite{MFTIne}:
\begin{equation}
	n_{i} = \sqrt{N_{c}N_{v}}\exp\!\bigg[ - \frac{E_{g}}{2k_{B}T} \bigg],
\end{equation}
\begin{conditions}
	$E_{g}$ & ширина запрещенной зоны (ЗЗ) п/п.
\end{conditions}

\subsection{Примесные полупроводники}
Примесный полупроводник - это полупроводник электрофизические свойства которого определяются, в основном, примесями других химических элементов.

Концентрация электронов в ЗП примесного п/п ~\cite{MFTIne}:
\begin{equation}
	n = \frac{N_{D}}{2}\bigg( 2 + \frac{1}{2}\bigg( \frac{2n_{i}}{N_{D}} \bigg)^{2} \bigg),
\end{equation}
\begin{conditions}
	$N_{D}$ & концентрация атомов легирующей примеси.
\end{conditions}