\section{Токоперенос через ГС}
Один из способов расчета плотности тока через гетероструктуру~--- это формула Цу--Есаки:
\begin{equation}
	\label{eq:J}
	J(V) = \frac{2mek_{B}T}{(2\pi)^{2}\hbar^{3}}\int\limits_{0}^{\infty}T(E)D(E)dE,
\end{equation}
\begin{conditions}
	$J(V)$ & плотность тока при приложенном напряжении $V$;\\
	$T(E)$ & прозрачность гетероструктуры;\\
	$D(E)$ & функция снабжения электронами;\\
	$m$ & эффективная масса электрона;\\
	$e$ & заряд электрона;\\
	$T$ & температура;\\
	$\hbar$ & постоянная Дирака;\\
	$k_{B}$ & постоянная Больцмана.
\end{conditions}
\begin{equation}
	D(E) = \ln\frac{1 + \exp{\frac{E_{F}-E}{k_{B}T}} }{ 1 + \exp{\frac{E_{F}-E-eV}{k_{B}T}} },
\end{equation}
\begin{conditions}
	\label{eq:D}
	$E_{F}$ & уровень Ферми;\\
	$V$ & приложенное напряжение.
\end{conditions}

Коэффициент прозрачности гетероструктуры определяется   как   отношение потока вероятности прошедших  через  структуру электронов  в  правом резервуаре к  падающим  на  неё  электронам  в  левом  резервуаре.  Поток вероятности находится из формулы:
\begin{equation}
	\label{eq:jP}
	\overline{j} = \frac{i\hbar}{2m}(\psi\nabla\psi^{*} - \psi^{*}\nabla\psi),
\end{equation}
\begin{conditions}
	$\psi$ & волновая функция электрона.
\end{conditions}
Будем рассматривать электроны, приходящие из левого контакта. Левому контакту соответствуют волновые функции $\psi_{L}$, а в правому~---  $\psi_{R}$.
\begin{gather}
	\label{eq:psiL}
	\psi_{L} = \exp [ik_{L}z];\\
	\label{eq:psiR}
	\psi_{R} = T_{L}\psi_{L} = T_{L}\exp [ik_{L}z],
\end{gather}
\begin{conditions}
	$T_{L}$ & Амплитуда прошедшей волновой функции;\\
	$k_{L}$ & волновой вектор в левом резервуаре.
\end{conditions}
Тогда коэффициент туннельной прозрачности:
\begin{equation}
	T(E) = |T_{L}|^{2}\frac{|k_{R}|m_{L}}{|k_{L}|m_{R}},
\end{equation}
\begin{conditions}
	$k_{R}$ & волновой вектор в правом резервуаре;\\
	$m_{R}$ & эффективная масса электрона в правом резервуаре;\\
	$m_{L}$ & эффективная масса электрона в левом резервуаре.
\end{conditions}
\subsection{Уравнение Шредингера}
Для нахождения волновых функций необходимо решить уравнение Шредингера. Для твердого тела уравнение Шредингера имеет вид:
\begin{gather*}
	-\frac{\hbar}{2}\Bigg[ \bigg( \frac{1}{m}\sum\limits_{i}\Delta_{i} + \sum\limits_{i}\frac{\Delta_{i}}{M_{i}} \bigg) + \frac{1}{2}\sum\limits_{i, j\neq i}\frac{e^{2}}{k_{k}|\overline{r_{i}} - \overline{r_{j}}|} + \frac{1}{2}\sum\limits_{i, j\neq i}\frac{Z_{i}Z_{j}e^{2}}{k_{k}|\overline{R_{i}} - \overline{R_{j}}|} -\\ - \frac{1}{2}\sum\limits_{i, j\neq i}\frac{Z_{i}e^{2}}{k_{k}|\overline{R_{i}} - \overline{r_{j}}|} \Bigg]\psi = E\psi
\end{gather*}
\begin{conditions}
	$M_{i}$ & масса атомного остова;\\
	$k_{k}$ & постоянная Кулона;\\
	$Z $& атомное число;\\
	$\frac{\Delta_{i}}{m}$ & кинетическая энергия $i$-ого электрона;\\
	$\frac{\Delta_{i}}{M_{i}}$ & кинетическая энергия $i$-ого атомного остова;\\
	$\frac{e^{2}}{k_{k}|\overline{r_{i}} - \overline{r_{j}}|}$ & потенциальное взаимодействие $i$ и $j$ электрона;\\
	$\frac{Z_{i}Z_{j}e^{2}}{k_{k}|\overline{R_{i}} - \overline{R_{j}}|}$ & потенциальное взаимодействие остовов;\\
	$\frac{Z_{i}e^{2}}{k_{k}|\overline{R_{i}} - \overline{r_{j}}|}$ & потенциальное взаимодействие остова и электрона.
\end{conditions}

Ряд приближений упрощает полное уравнение Шредингера для твердого тела:
\begin{enumerate}
	\item Атомные остовы находятся в состоянии покоя;
	\item Электрон движется, не взаимодействуя с другими электронами, в некотором эффективном поле, создаваемым остальными электронами;
	\item Движение электрона в периодическом потенциале заменяется на эффективную массу.
\end{enumerate}
Упрощенное уравнение Шредингера:
\begin{equation}
	\label{eq:ShredGen}
	-\frac{\hbar^{2}}{2m}\Delta\psi + U\psi = E\psi,
\end{equation}
\begin{conditions}
	$ U $ & потенциальный профиль.
\end{conditions}
Одномерное ур. Шредингера:
\begin{equation}
	\label{eq:Shred}
	-\frac{\hbar^{2}}{2m}\frac{d^{2}}{dz^{2}}\psi(z) + U(z)\psi(z) = E\psi(z).
\end{equation}
Для решения уравнения на границе гетероперехода рассматриваются условия непрерывности волновой функции и непрерывности потока плотности вероятности~--- эти условия так же называются условием Бастарда:
\begin{equation}
	\label{eq:Bastard}
	\begin{cases}
		\psi_{I} = \psi_{II};\\
		\frac{1}{m_{I}}\frac{d}{dz}\psi_{I} = \frac{1}{m_{II}}\frac{d}{dz}\psi_{II},
	\end{cases}
\end{equation}
\begin{conditions}
	$m_{I}$& эффективная масса в $I$ структуре;\\
	$m_{II}$& эффективная масса во $II$ структуре;\\
	$\psi_{I}$& волновая функция в $I$ структуре;\\
	$\psi_{II}$& волновая функция во $II$ структуре.
\end{conditions}
Учтем эффективную массу в ур.~\ref{eq:Shred}:
\begin{equation}
	\label{eq:ShredM}
	-\frac{\hbar^{2}}{2}\frac{d}{dz}\frac{1}{m(z)}\frac{d}{dz}\psi(z) + U(z)\psi(z) = E\psi(z).
\end{equation}
В случае произвольного потенциального рельефа для решения уравнения Шредингера применяются численные методы.

