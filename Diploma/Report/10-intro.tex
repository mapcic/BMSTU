\Introduction
Объектом исследования данной работы являются GaAs--гетероструктуры и приборы на их основе, в частности --- резонансно--туннельные диоды. 

Предмет исследования --- методы моделирования диффузионного размытия гетероструктур на основе GaAs и токопереноса в таких структурах.

\textbf{Цель работы:}
\begin{itemize}
	\item Разработка алгоритма прогнозирования деградации приборов на основе $GaAs$ гетероструктур.
	% \item Разработка модели термической деградации слоистых гетероструктур на основе GaAs для интеграции в методику обеспечения заданного уровня надёжности устройства на их основе.
\end{itemize}

\textbf{Задачи работы:}
\begin{itemize}
	% \item Исследование математического аппарата для моделирования диффузионного размытия гетероструктур под действием градиента концентрации при фиксированной температуре системы;
	% \item Исследование математического аппарата для моделирования токопереноса через гетероструктуру;
	% \item Разработка алгоритма термической деградации гетероструктуры на основе $GaAs$.
	\item Моделирование диффузионного размытия гетероструктур на основе $GaAs$ под действием градиента концентрации при фиксированной температуре;
	\item Моделирование токопереноса через гетероструктуру на основе $GaAs$;
	\item Разработка алгоритма деградации резонансно-туннельной гетероструктуры на основе $GaAs$.
\end{itemize}

Устройства на гетероструктурах (далее ГС) открыли новый виток в электронике. Данная сфера активно используется и развивается. Особенность работы таких устройств объясняется различной зонной структурой материалов, используемых в ГС. На границе ГС из-за различного химического состава неизбежно начинается взаимодиффузия атомов и зонная структура изменяется --- деградирует, причем скорость деградации зависит как от химического состава, так и от внешних условий. 

Разработка новых устройств на основе $GaAs$ и подобных  материалах требует необходимости прогнозирования изменения функциональных параметров приборов на их основе под действием внешней среды и времени. Именно поэтому необходим такой программный продукт, с помощью которого можно было бы исследовать деградацию ГС произвольной конфигурации. Далее в работе исследуется термическая деградация резонансно-туннельной гетероструктуры (РТГС), которая свойственна ГС при изготовлении и эксплуатации.

Факторы влияющие на деградацию GaAs гетероструктуры:
\begin{enumerate}
	\item Технологические: \begin{enumerate}
		\item МЛЭ (Для РТД $t\approx560$ секунд, $T\approx550$--$650^{\circ}$C);
		\item Отжиг (Для РТД $t\approx30$ секунд, $T\approx800^{\circ}$C);
		\item Отжиг металлических контактов (Для РТД $t\approx15$--$120$ секунд, $T\approx350$--$470^{\circ}$C).
	\end{enumerate}
	\item Эксплуатационные:
	\begin{enumerate}
		\item Температурная и временная нагрузка ($t\approx 10$ лет, $T\approx25$--$100^{\circ}$C).
	\end{enumerate}
\end{enumerate}
