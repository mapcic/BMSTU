\section{Деградация приборов на основе гетероструктур}
Деградация~--- процесс ухудшения характеристик какого-либо объекта c течением времени.

Изучая деградацию ГС рассматривают следующие параметры:
\begin{itemize}
	\item Вольт-амперная характеристика (ВАХ);
	\item Высота потенциального барьера (ПБ);
	\item Ширина потенциального барьера;
	\item Ширина потенциальной ямы (ПЯ);
	\item Т.д...
\end{itemize}

ГС используют для построения резонансно-туннельный диод (РТД), квантовых точек (КТ), транзисторов с высокой подвижностью электронов (HEMT) и так далее. 

Химический состав ГС определяет ее зонную структуру, из чего вытекают особенности работы тех или иных устройств на ГС.

Одна из причин деградации ГС~--- диффузионное размытие ГП, вызванное: 
\begin{itemize}
	\item Градиентом температуры; 
	\item Градиентом концентрации; 
	\item Градиентом давления; 
	\item и т.д... 
\end{itemize}

Диффузионное размытие под действием градиента концентрации описывается с помощью законов Фика.

\section{Диффузия}
Диффузия — это обусловленный хаотическим тепловым движением перенос атомов, он может стать направленным под действием градиента концентрации или температуры.

Диффундировать могут как собственные атомы решетки, так и атомы растворенных в полупроводнике элементов, а также точечные дефекты структуры кристалла — междоузельные атомы и вакансии.

\subsection{Законы Фика}
Первый закон Фика говорит, что плотность потока вещества пропорциональна коэффициенту диффузии ($D$) и градиенту концентрации ($C$). Является стационарным уравнением.
\begin{gather}
	\overline{J} = - D \nabla C;\\
	\overline{J}_{x} = - \overline{e}_{x}D_{x} \frac{\delta}{\delta x} C_{x}.\\
\end{gather}

Второй закон Фика связывает пространственное и временное изменения концентрации.
\begin{gather}
	\frac{\delta}{\delta t}C = - \nabla (D \nabla C);\\
	\frac{\delta}{\delta t}C_{x} = - \frac{\delta}{\delta x} D_{x} \frac{\delta}{\delta x} C_{x}.
\end{gather}

\subsection{Механизмы диффузии}

Вакансионный механизм диффузии — заключается в миграции атомов по кристаллической решётке при помощи вакансий.

Межузельный механизм диффузии — заключается в переносе вещества межузельными атомами.

Прямой обмен атомов местами — заключается в том, что два соседних атома одним прыжком обмениваются местами в решетке кристалла.

\subsection{Коэффициент диффузии}
% Коэффициент диффузии в терминах случайных блужданий можно записать(для простой кубической решетки):
% \begin{equation}
% 	D = \frac{1}{6}\lambda^{2}\nu,
% \end{equation}
% \begin{conditions}
% 	$\lambda$ & расстояние между соседними кристаллографическими плоскостями;\\
% 	$\nu$ & частота скачков диффундирующих атомов.
% \end{conditions}

% Частота скачков $\nu$ зависит от температуры ($T$)
% \begin{equation}
% 	\nu = \nu_{0}\exp\bigg[-\frac{E_{a}}{k_{B}T}\bigg],
% \end{equation}
% \begin{conditions}
% 	$E_{a}$ & энергия активации;\\
% 	$k_{B}$ & константа Больцмана;\\
% 	$T$ & температура атома;\\
% 	$\mu_{0}$ & константа.
% \end{conditions}

Коэффициент диффузии ($D$)~--- макроскопическая величина, которая определяется экспериментально. Коэффициент диффузии зависит от температуры(T) по закону Аррениуса:
\begin{equation}
	D = D_{0}\exp\bigg[-\frac{E_{a}}{k_{B}T}\bigg],
\end{equation}
\begin{conditions}
	$D_{0}$ & предэкспоненциальный множитель.
\end{conditions}
Коэффициент ($D_{0}$) и энергия активации ($E_{a}$) не зависят от температуры.

% \subsection{Коэффициент диффузии $Al$, $Si$ в $GaAs$}
Основным механизмом диффузии $Al$ и $Si$ в $GaAs$ является диффузия по вакансиям галлия ($V_{Ga}$). Это связано с тем, что атомы $Al$ и $Si$ имеют сходные массы и размеры. 

С учетом эффекта уровня Ферми коэффициент диффузии $Al$ и $Si$ в $GaAs$ получен в работах ~\cite{getMeshkov}, ~\cite{Meshkov}, ~\cite{Meshkov}, ~\cite{Makeev}:
\begin{equation}
	D_{Al,Si} = D_{i-GaAs}\Big( \frac{N_{D}}{n_{i}} \Big)^{3} = D_{0}\exp\bigg[-\frac{3.5}{k_{B}T}\bigg]\Big( \frac{n}{n_{i}} \Big)^{3},
\end{equation}
\begin{conditions}
	$n$ & концентрация донорной примеси ($Si$);\\
	$n_{i}$ & концентрация собственных носителей заряда.
\end{conditions}

Концентрация собственных носителей заряда ~\cite{MFTIne}:
\begin{gather}
	n_{i} = \sqrt{N_{c}N_{v}}\exp\!\bigg[ - \frac{E_{g}}{2k_{B}T} \bigg];\\
	N_{c} = 2\Big[ \frac{2\pi m_{e}^{\ast}k_{B}T}{h^{2}} \Big]^{\frac{3}{2}};\\
	N_{v} = 2\Big[ \frac{2\pi m_{h}^{\ast}k_{B}T}{h^{2}} \Big]^{\frac{3}{2}},
\end{gather}
\begin{conditions}
	$E_{g}$ & ширина запрещенной зоны (ЗЗ) п/п.
\end{conditions}



% \section{Полупроводники}
% Полупроводники (п/п)- широкий класс веществ, в которых концентрация подвижных носителей заряда значительно ниже, чем концентрация атомов, и может изменяться под влиянием температуры. освещения или относительно малого количества примесей.

% Эффективная плотность состояний в зоне проводимости (ЗП)~\cite{MFTIne}:
% \begin{equation}
% 	N_{c} = 2\Big[ \frac{2\pi m_{e}^{\ast}k_{B}T}{h^{2}} \Big]
% 	^{\frac{3}{2}},
% \end{equation}
% \begin{conditions}
% 	$m_{e}^{\ast}$ & эффективная масса электрона;\\
% 	$k_{B}$ & константа Больцмана;\\
% 	$h$ & постоянная Планка;\\
% 	$T$ & температура.
% \end{conditions}

% Эффективная плотность состояний в валентной зоне (ВЗ)~\cite{MFTIne}:
% \begin{equation}
% 	N_{v} = 2\Big[ \frac{2\pi m_{h}^{\ast}k_{B}T}{h^{2}} \Big]
% 	^{\frac{3}{2}},
% \end{equation}
% \begin{conditions}
% 	$m_{h}^{\ast}$ & эффективная масса дырки.
% \end{conditions}

% \subsection{Собственные полупроводники}
% Собственный полупроводник (п/п i-типа)~--- это чистый полупроводник, содержание посторонних примесей в котором не превышает $10^{−8} … 10^{−9}\%$.

% Концентрация собственных носителей заряда в ЗП~\cite{MFTIne}:
% \begin{equation}
% 	n_{i} = \sqrt{N_{c}N_{v}}\exp\!\bigg[ - \frac{E_{g}}{2k_{B}T} \bigg],
% \end{equation}
% \begin{conditions}
% 	$E_{g}$ & ширина запрещенной зоны (ЗЗ) п/п.
% \end{conditions}

% \subsection{Примесные полупроводники}
% Примесный полупроводник - это полупроводник электрофизические свойства которого определяются, в основном, примесями других химических элементов.

% Концентрация электронов в ЗП примесного п/п ~\cite{MFTIne}:
% \begin{equation}
% 	n = \frac{N_{D}}{2}\bigg( 2 + \frac{1}{2}\bigg( \frac{2n_{i}}{N_{D}} \bigg)^{2} \bigg),
% \end{equation}
% \begin{conditions}
% 	$N_{D}$ & концентрация атомов легирующей примеси.
% \end{conditions}

% % \section{Деградация}
% % Деградация~--- процесс ухудшения характеристик какого-либо объекта или явления с течением времени, постепенное ухудшение, упадок, снижение качества.

% % Изучая деградацию ГС рассматривают следующие параметры:
% % \begin{itemize}
% % 	\item Вольт-амперная характеристика (ВАХ);
% % 	\item Высота потенциального барьера (ПБ);
% % 	\item Ширина потенциального барьера;
% % 	\item Ширина потенциальной ямы (ПЯ);
% % 	\item Т.п...
% % \end{itemize}

% % Все параметры ГС тесно связаны с друг с другом. Изменение одного влечет изменение остальных.

% % Один из примеров ГС~--- это резонансно-туннельный диод (РТД). РТД используют в качестве преобразователя частот в смесителях, где преобразование частот зависит от формы ВАХ РТД, которая подвержена деградации.

% % Форма ВАХ задается ГС активной области РТД. Исследование и моделирование деградации ВАХ ГС важная задача. В зависимости от отрасли необходимо гарантировать различную $T_{\gamma=0.999}$.

% % Одна из причин деградации ВАХ ГС~--- диффузионное размытие профиля дна зоны проводимости ($E_{c}$).