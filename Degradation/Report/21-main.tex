% \section{Гетероструктура}
% Гетероструктура~--- полупроводниковая структура с несколькими гетеропереходами (ГП). 

% ГС получили широкое распространение из-за возможности, изменяя на границах ГС ширину запрещённой зоны, управлять движением носителей заряда.

% Гетеропереход~--- контакт двух различных по химическому составу монокристаллических или аморфных полупроводников.

% ГП может образоваться между полупроводниками с абсолютно одинаковыми постоянными решетки, образующими монолитный, однородный в контакте, кристалл.
% \begin{enumerate}
% 	\item $GaAs$--$AlAs$;
% 	\item $GaN$--$AlN$;
% 	\item $GaSb$--$AlSb$--$InAs$;
% 	\item $GaAs$--$Ge$.
% \end{enumerate}

% \subsection{Зонная диаграмма гетероперехода}
% Для построения зонной диаграммы необходимо знать ширину запрещенной зоны ($E_{g}$) и положение уровня Ферми ($E_{F}$) для контактируемых полупроводников.

% \begin{figure}[h]
%   \centering
%   \includegraphics[width=.5\linewidth]{assets/Eg}
%   \caption{Зонная диаграмма перехода между полупроводниками с различными $E_{g}$}
%   \label{img:2.0.0}
% \end{figure}

% Одна из самых распространенных ГС~--- это ГС на основе твердого раствора $Al_{x}Ga_{1−x}As$, где $x$ -- это доля замещения.
% Основные характеристики $Al_{x}Ga_{1−x}As$:

% \begin{center}
%   \begin{longtable}{|c|c|}
%     \caption{Основные параметры $Al_{x}Ga_{1−x}As$}
%     \label{tab:2.0.0}
%     \\ \hline
%     Параметр & $Al_{x}Ga_{1−x}As$ \\
%     \hline \endfirsthead
%     \subcaption{Продолжение таблицы~\ref{tab:2.0.0}}
%     \\ \hline \endhead
%     \hline \subcaption{Продолжение на след. стр.}
%     \endfoot
%     \hline \endlastfoot
% 	Кристаллическая структура& Типа цинковой обманки \\ \hline
% 	Постоянная решетки $a[nm]$  & $0.56533+0.00078x$ \\ \hline
% 	$E_{g}^{\Gamma}[eV],\, x < 0.45$    & $1.424+1.247x$ \\ \hline
% 	$E_{g}^{\Gamma}[eV],\, x > 0.45$    & $1.656+0.215x+0.143x^{2}$ \\ \hline
% 	% $\Delta E_{c}^{\Gamma}[eV],\, x < 0.45$    & $0.773x$ \\ \hline
% 	% $\Delta E_{c}^{\Gamma}[eV],\, x > 0.45$    & $0.232-0.259x+1.147x^{2}$ \\ \hline
% 	$m_{e}^{\Gamma}$    & $0.067+0.083x$ \\ \hline
% 	$m_{lh}$    & $0.082+0.071x$ \\ \hline
% 	$N_{atoms}[1/sm^{-3}]$    & $(4.42-0.17x)10^{22}$
%   \end{longtable}
% \end{center}

% Следует также принимать во внимание, что полупроводники могут иметь минимумы зоны проводимости в разных точках зоны Брюллиена. К примеру, минимум зоны проводимости $GaAs$ находится в точке $\Gamma$, в то время как наименьший минимум в $AlAs$ близок к точке $X$. Таким образом, природа низшего минимума зоны проводимости меняется при изменении доли $Al$ в твердом растворе $Al_{x}Ga_{1−x}As$. Низший минимум в $Al_{x}Ga_{1−x}As$ изменяется от прямого расположения (минимум в $\Gamma$) зон до непрямой зонной структуры (минимум в $Х$) при содержании $Al \approx 45\%$. Обычно твердый раствор $Al_{x}Ga_{1−x}As$ получают с долей $Al$, меньше $0.45$, чтобы получить прямое расположение зон.

% Рассматривая ГП $i\!-\!GaAs/i\!-\!Al_{x}Ga_{1−x}As$, при $x < 0.45$, получим высоту потенциальной ступеньки в зоне проводимости ($E_{c2} - E_{c1}$) $U \approx 1.247*x\, eV$.
% % \section{Конкретная структура на основе $Al_{x}Ga_{1−x}As$}

% % Более подробно рассмотрим гетероструктуру на основе $Al_{x}Ga_{1−x}As$. Возьмем $GaAl$/$AlAs$/$GaAl$/$AlAs$/$GaAl$, с соответствующими размерами $8$/$4$/$6$/$4$/$8$ -- монослоев.
\subsection{Резонансно--туннельная гетероструктура}
Резонансно--туннельная гетероструктура (РТГС) --- это ГС, в некоторой области которой электронный газ (ЭГ) перестает быть трехмерным свободным ЭГ (3D ЭГ), а становится становится двумерным ЭГ (2D ЭГ).

Структурная схема РТГС представлена на рис.\ref{img:rtgs}
\begin{figure}[h]
  \centering
  \includegraphics[width=.6\linewidth]{assets/rtgs}
  \caption{Структурная схема устройства с РТГС}
  \label{img:rtgs}
\end{figure}

Омический контакт выполняет роль связи РТГС с электрической схемой. Величина сопротивления искажает форму ВАХ РТГС, чем оно меньше, тем лучше.

Приконтактная область снабжает РТГС основными носителями заряда. Область сильно легируется основными носителями заряда до вырожденного состояния. Размеры приконтактных областей побираются так, чтобы концентрация основных носителей заряда приходила к равновесной на их концах.

Спейсеры изготавливаются из чистого полупроводника и предохраняют барьер и яму от проникновения туда легирующей примеси, так же спейсер препятствует накоплению заряда вблизи и внутри ямы.

Барьеры и яма формируют 2D ЭГ. Величина барьера и ямы влияют на положение резонансного уровня, прозрачность РТГС и т.д.

\subsection{Резонансно--туннельная гетероструктура на основе $Al_{x}Ga_{1-x}As$}.
РГТС на основе $Al_{x}Ga_{1-x}As$ показана на рис.\ref{img:rtgsAlGaAs}
\begin{figure}[h]
  \centering
  \includegraphics[width=0.99\linewidth]{assets/rtgsAlGaAs}
  \caption{Структурная схема устройства с РТГС на основе $Al_{x}Ga_{1-x}As$}
  \label{img:rtgsAlGaAs}
\end{figure}

Зонная структура такой РТГС представлена на рис.\ref{img:DifCloseAlGaAs}. Меняя процентное содержание $Al$ в $Al_{x}Ga_{1-x}As$, мы можем получать необходимую высоту барьеров.

\begin{figure}[h]
  \centering
  \includegraphics[width=0.99\linewidth]{assets/DifCloseAlGaAs}
  \caption{Зонная структура РТГС на основе $Al_{x}Ga_{1-x}As$}
  \label{img:DifCloseAlGaAs}
\end{figure}